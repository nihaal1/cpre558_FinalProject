\documentclass[twocolumn, 10pt]{article}
\usepackage[utf8]{inputenc}
\usepackage{geometry}
\usepackage{graphicx}
\usepackage{hyperref}
\usepackage{biblatex}
\addbibresource{references.bib}

\geometry{a4paper, margin=1.0in}

\title{\textbf{Project Plan: Energy-Efficient Real-Time Scheduling in Sensor Networks via DVFS on Raspberry Pi}}
\author{
    Nihaal Zaheer \\
    \texttt{nzaheer@iastate.edu} \\
    CprE 558 Section 2
    \and
    Varun Advani \\
    \texttt{vsadvani@iastate.edu} \\
    CprE 558 Section 2
}
\date{}

\begin{document}

\maketitle

\section*{Project Type}
\vspace{-2mm}
Implementation (Type 2) - This project focuses on the implementation and testing of various DVFS algorithms in a Raspberry Pi-based sensor network environment to optimize energy consumption.

\vspace{-3mm}

\section*{Project Goal}
\vspace{-2mm}
The project aims to explore the implementation of Dynamic Voltage and Frequency Scaling (DVFS) techniques/algorithms and explore scalability of the Worst Case Execution Time (WCET) w.r.t the processor speed within an operating system running on a Raspberry Pi, as part of a sensor network. The goals include:

\begin{itemize}
    \item \textbf{OS-related Focus:} Integrate DVFS techniques with the Raspberry Pi's Linux-based operating system to manage CPU speed and voltage dynamically, based on real-time computational demands from sensor data processing.
    \item \textbf{Architecture-related Focus:} Analyze the impact of DVFS on the Raspberry Pi's hardware architecture, particularly on CPU and memory performance, power consumption, and thermal dynamics.
    \item \textbf{Protocol-related Focus:} Develop and test protocols for efficient communication between the Raspberry Pi and the sensors, ensuring data integrity and timely processing while under different DVFS states.
    \item \textbf{Application-related Focus:} Implement a real-time environmental monitoring application or intrudision detection application, that utilizes sensor data to demonstrate the practical effectiveness of DVFS in managing energy consumption while maintaining data processing accuracy and speed.
\end{itemize}

\vspace{5mm}

\section*{Solution Approach}
\vspace{-2mm}
Our approach to implementing energy-efficient real-time scheduling in a Raspberry Pi-based sensor network focuses on the application of Dynamic Voltage and Frequency Scaling (DVFS) techniques. The approach encompasses both hardware and software aspects, involving the following key components:

\vspace{-2mm}

\subsection*{Hardware and Software Setup}
\begin{itemize} 
    \item A Raspberry Pi 3 B+ with an ARM cortex 
  A53 processor.
    \item $PICAN2$\\
    \url {https://copperhilltech.com/pican-2-can-bus-interface-for-raspberry-pi/}
    \item $DHT11$ Temperature/Humidity Sensor
    \url{https://www.adafruit.com/product/386}
    \item $KY038$ Microphone Sound Sensor
    \url{https://sensorkit.joy-it.net/en/sensors/ky-038}
    \item $BMP085$ Barometric Sensor ($300-1100 hPa$)
    \url {https://www.adafruit.com/product/391#technical-details}
    \item Optional PIR Motion Sensor 
    \url {https://www.adafruit.com/product/189}
    \item Optional USB interfaced webcam for object detection (OpenCV)
    \item Optional power measurement tools connected to the Raspberry Pi, for precise monitoring of power consumption during different DVFS states.
     \item The Raspberry Pi OS image will be customized with the integration of the PREEMPT RT patch for enhanced real-time capabilities. The setup guidelines will be followed as per \url{https://robskelly.com/2020/10/14/raspberry-pi-4-with-64-bit-os-and-preempt_rt/}.
     \vspace{3mm}
    \item \textbf{Sensor Data Management:} Development of Python or C++ scripts for efficient sensor data collection, aggregation, and pre-processing. Use of libraries like RPi.GPIO, PySerial, and others for interfacing with different sensors.
    \item \textbf{DVFS Control and Monitoring:} Installation and configuration of CPU frequency scaling tools like cpufrequtils or cpupower for managing and monitoring the CPU's operational states.
    \item \textbf{Real-time Data Analysis and Reporting:} Utilization of lightweight databases or data handling tools like SQLite or Pandas (Python) for real-time data logging and analysis.
    
   
\end{itemize}


\subsection*{DVFS Techniques}
\begin{itemize}
    \item Implementing and testing different DVFS algorithms to find an optimal balance between energy efficiency and performance. These include:
    \begin{itemize}
        \item \textbf{Ondemand Governor:} Dynamically adjusts the CPU frequency according to current load, increasing frequency during high-load periods and decreasing during low-load periods.
        \item \textbf{Conservative Governor:} Similar to the ondemand governor, but with a more gradual adjustment of CPU frequency, aiming for a more stable performance and energy usage.
        \item \textbf{Custom Algorithm:} Tailored to sensor network data characteristics, adjusting frequency and voltage based on predictive models of sensor data loads and processing requirements.
    \end{itemize}
\end{itemize}

\subsection*{Software Development}
\begin{itemize}
    \item Developing a lightweight, efficient operating system configuration on the Raspberry Pi, optimized for low-power operation and quick sensor data processing.
    \item Writing scripts and applications in Python or C++ for sensor data collection, processing, and DVFS state control.
\end{itemize}


\subsection*{Performance and Energy Consumption Monitoring}
\begin{itemize}
    \item Systematic logging of CPU frequency, voltage levels, sensor data throughput, and energy consumption under various DVFS modes.
    \item Analysis tools developed to compare and evaluate the energy-performance trade-offs for different DVFS strategies.
\end{itemize}


This approach integrates technical elements from both the hardware and software realms, aiming to deliver a comprehensive solution that addresses the challenges of energy efficiency in a real-time sensor network environment.
\vspace{-2mm}
\section*{Expected Outcomes}
\vspace{-2mm}
The implementation of energy-efficient real-time scheduling in sensor networks using DVFS on Raspberry Pi aims to achieve the following outcomes:

\subsection*{Performance Evaluation}
\begin{itemize}
    \item Detailed performance metrics comparing the implemented DVFS algorithms (ondemand, conservative, and custom) against standard operating conditions without DVFS. This comparison will focus on:
    \begin{itemize}
        \item Energy consumption reduction percentages.
        \item Sensor data processing latency.
        \item System responsiveness under different load scenarios.
    \end{itemize}
    \item Analysis of the trade-offs between energy efficiency and data processing performance, providing insights into the practical implications of DVFS in real-world IoT environments.
\end{itemize}

\subsection*{Prototype Demonstration}
\begin{itemize}
    \item A working prototype of the Raspberry Pi with connected sensors, demonstrating the practical application of DVFS techniques in managing energy consumption while maintaining data integrity and processing speed.
    \item Demonstration of how the system adapts to varying environmental data loads and how DVFS algorithms effectively modulate power usage.  
\end{itemize}

\subsection*{Documentation and Insights}
\begin{itemize}
    \item Comprehensive documentation covering the development process, DVFS implementation strategies, and findings from performance evaluations.
    \item Valuable insights and recommendations for future work in the domain of energy-efficient real-time scheduling in sensor networks.
\end{itemize}

\section*{List of References}
\nocite{*}  
\printbibliography[heading=none]  





\end{document}